%============================================================================
\documentclass[12pt]{article}
\listfiles
\usepackage[T1]{fontenc}
\usepackage{fourier}
\usepackage{latexsym}
\usepackage{epsfig}
\usepackage{amsfonts}
\usepackage{amsmath}
\usepackage{amssymb}
\usepackage[usenames,dvipsnames]{color}
\usepackage{ifthen}
\usepackage{hyperref}
\usepackage{subfigure}
%Formatting

% 1in margins (ignoring page number in footer)
\setlength{\textwidth}{6.5in}
\setlength{\textheight}{9in}
\setlength{\hoffset}{0in}
\setlength{\voffset}{0in}
\setlength{\oddsidemargin}{0in}
\setlength{\evensidemargin}{0in}
\setlength{\topmargin}{0in}
\setlength{\headheight}{0in}
\setlength{\headsep}{0in}
\setlength{\footskip}{20pt}
\setlength{\parskip}{0.1in}
\setlength{\parindent}{0.0in}
%============================================================================

%============================================================================
% Begin document 
%
\begin{document}

\begin{center}
{\Large{\bf Contributions of magnetized laboratory experiments to fundamental plasma physics}} \\[0.1truein]
Troy Carter (UCLA), Cary Forest (UW Madison)
\end{center}

\hrule


This review article will cover important contributions by foundational
plasma physics experiments.  By foundational, we mean that there is a 
uniquely defined process that can be isolated and studied, and at some level can be 
isolated from the complexity of a system (eg. fusion devices, astrophysical plasmas, etc).
While individual physics topics have been reviewed before (e.g. magnetic reconnection, drift
wave turbulence), to our knowledge, this is the first review article
of its kind.  For this reason, there is potentially a large body of
work to cover. To focus the article, we only provide a broad overview of the history of
foundational experiments prior to 1990, and will limit the detailed review to
laboratory studies of magnetized plasmas and in roughly the last
two decades of work. The goal of the article is
to highlight important advances in our understanding of fundamental
plasma physics made possible by laboratory experiments.  A focus will
be given to experiments done in non-fusion experiments, however
advances made in fundamental physics through using confinement
experiments (e.g. tokamaks and reversed-field pinches) will, when appropriate, be included.
In addition, future opportunities in laboratory plasma physics would be
discussed. An potential list of topics for the article is given below.


%% start, perhaps, with a description of the sorts of tools (plasmas and devices) which are now
%% being broadly used for basic experiments.

%% Devices:  linear, toroidal, helical, emphasize the development of the ACT-1/CDX-U type of device--> now used in Laussane, projected for ET, penning
%% Sources: rf, inductive, electrostatic discharges, spheromak guns ( merging)
%% Diagnostics:  


%% Hmm.  Can we add some meat to the idea of foundational.
%% Text books sort of provide foundations:   
%% \begin{itemize}
%%  \item single particle physics (e.g. fast particle physics on LAPD,stochastic transport, ..., ECH startup experiments)
%%   \item kinetic effects (LIF experiments)
%%   \item two-fluid  
%%   \item MHD 
%%   \item Waves (EBWs)
%%   \item Essentially non-linear phenomena (turbulence, sheaths, ..., probe theory??)
%%   \end{itemize}


%% The following sort of shows where the action has been:

\begin{description}
\item[Introduction, overview]  Discuss types of experiments,
  plasma sources, diagnostics. 
\item[Fundamental processes, collisional transport]  Experimental
  studies of collisional transport heat, particles and momentum in
  magnetized laboratory plasmas.
\item[Waves, instabilities, flows, turbulence]  This
  section will include a discussion of the linear and nonlinear
  physics of plasma waves (e.g. Alfv\'{e}n waves), small-scale instabilities (e.g. drift
  waves), flow and flow-driven instabilities (zonal flows,
  Kelvin-Helmholtz), coherent structures in turbulence
  (e.g. ``blobs''), transport due to turbulence in magnetized
  plasmas. 
\item[Non-neutral plasmas] Stability and transport in magnetized non-neutral
  (e.g. pure electron) plasmas, autoresonance, antimatter production.
\item[Reconnection]  Sweet-Parker and fast reconnection in the
  laboratory, turbulence in current sheets, dissipation, ion heating.
\item[Magnetically driven phenomena: jets, flux ropes, macroscopic
  instabilities]  Kink instability, line-tying, stability and
  interaction of magnetic flux ropes, magnetic flux conversion and flux transport, dynamo
\item[Conclusion, future opportunities] Future opportunities in these areas will be discussed: what problems
remain and what types of experiments would be needed to address these
problems. 

\end{description}





\end{document}
